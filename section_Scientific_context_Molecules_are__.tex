\section{Scientific context} 


Molecules are present everywhere in the interstellar medium, even in the diffuse medium where the UV field was believed to destroy them (see for instance \cite{2012ApJ...753L..28L}). In the densest and coldest parts of the ISM (i.e. cold cores), shielded from the interstellar UV field, many molecules are formed and can survive. In addition to large non organic species such as cyanopolyynes (HC$_n$N with $n$ up to to 11), it was recently found that complex organic molecules (COMs) are formed in these cold conditions and are observable in the gas phase (\cite{Bacmann_2012}, \cite{2014ApJ...795L...2V}). \textbf{To what extend then molecular complexity starts in the dense cores that form stars and planets and what are the key processes involved in the formation of these species?} To answer these questions, we however need to understand the formation of the more simple species as they are the precursors of these COMs. \\
The formation of interstellar species in the gas-phase has been studied for a long time. Chemical networks have been built to index all individual reactions and associated rate coefficients that play a role in the determination of the molecular abundances (Wakelam et al. 2010, Smith 2011). Since 2009, with a group of chemists, we have been working on gas-phase networks in order to propose to the community the most complete and accurate data to model the chemistry in the ISM. As an example, we studied in detail the chemistry of nitrogen bearing species (\cite{2013PCCP...1513888D}, \cite{2012PNAS..10910233D}, \cite{2014MNRAS.443..398L}, \cite{2013arXiv1310.4350W}). We have also strongly modified the networks for carbon bearing species (\cite{Wakelam_2009}, \cite{2014MNRAS.437..930L}, \cite{2015MNRAS.453L..48W}). Other chemical families are currently under review such as the sulphur-bearing species. All these results have been (or will be) put in the public on line database KIDA (http://kida.obs.u-bordeaux1.fr/).
In cold cores, the surface of interstellar grains play a crucial role in this chemistry. This role was believed for several years to be dominated by the sticking of gas-phase species onto the grain surfaces. At these very low temperatures, the diffusion of species on the surfaces and their subsequent reactions is assumed to be very limited. The thermal desorption of these species in the gas-phase is even less efficient. With the recent observation of COMs in the gas-phase in cold cores, such as HCOOCH$_3$, CH$_3$OCH$_3$ and CH$_3$CHO, this belief has been questioned. Indeed, these species were up to now associated with warm star forming regions where the temperature would allow for their formation on the grains and subsequent evaporation in the gas-phase (see \cite{Herbst_2009}). Now several scenario have been proposed to explain these observations and would need for testing to be validated (\cite{2013ApJ...769...34V}, \cite{2015MNRAS.449L..16B}, \cite{2015MNRAS.447.4004R}).



%\begin{itemize}
%\item{Cold core in the evolutionary context of the ISM}
%\item{Cold temperature chemistry: COMs ?}
%\item{Status of present work on chemistry in dense cores (TMC1, B1 ?), deep integrations of a few selected sources}
%\end{itemize}
%Need of systematic studies -> Survey
  
  
  
  
  
  
  
  
  
  
  
  
  
  
  
  