\section{Scientific context} 

Molecules are present everywhere in the interstellar medium, even in the diffuse medium where the UV field was believed to destroy them (see for instance \cite{2012ApJ...753L..28L}). In the densest and coldest parts of the ISM (i.e. cold cores), shielded from the interstellar UV field, many molecules are formed and can survive. The formation of these species in the gas-phase has been studied for a long time. Chemical networks have been built to index all individual reactions and associated rate coefficients that play a role in the determination of the molecular abundances. We now know that both ion-neutral reactions and barrier-free neutral-neutral reactions drive this chemistry (\cite{2010SSRv..156...13W}, \cite{Smith_2011}). Since 2009, with a group of chemists, we have been working on gas-phase networks in order to propose to the community the most complete and accurate data to model the chemistry in the ISM. Up to now, several families have been revisited. We studied in details the chemistry of nitrogen bearing species (\cite{2013PCCP...1513888D}, \cite{2012PNAS..10910233D}, \cite{2014MNRAS.443..398L}, \cite{2013arXiv1310.4350W}, in particular the reactions for N$_2$, HCN and HNC. We have also strongly modified the networks for carbon bearing species (\cite{Wakelam_2009}, \cite{2014MNRAS.437..930L}, \cite{2015MNRAS.453L..48W}). The main results of these studies were to decrease the predicted abundances of large carbon bearing species and propose a formation scenario for HCCO (recently observed in dense cores by \cite{Ag_ndez_2015}). We also made predictions on the atomic carbon abundance in these cores based on the HCN/HNC abundance ratio (AND OTHER?). Other families are currently being reviewed such as Sulphur-bearing molecules and C$_3$H$_X$ species. In the end, about 2000 reactions (i.e. a little less than half of gas-phase reactions in current networks) have been investigated. All these results have been (or will be) put in the public on line database KIDA (http://kida.obs.u-bordeaux1.fr/). \\
In cold cores, the surface of interstellar grains play a role in this chemistry. This role was believed for several years to be dominated by the sticking of gas-phase species onto the grain surfaces. At these very low temperatures, the diffusion of species on the surfaces and their subsequent reactions is assumed to be very limited. The thermal desorption of these species in the gas-phase is even less efficient. With the recent observation of complex organic molecules in the gas-phase in cold cores such as HCOOCH$_3$, CH$_3$OCH$_3$ and CH$_3$CHO (\cite{Bacmann_2012}, \cite{2014ApJ...795L...2V}), this belief has been questioned. Indeed, these species were up to now associated with warm star forming regions where the temperature would allow for their formation on the grains and subsequent evaporation in the gas-phase (see \cite{Herbst_2009}). Now several scenario have been proposed to explain these observations and would need for testing to be validated (\cite{2013ApJ...769...34V}, \cite{2015MNRAS.449L..16B}, \cite{2015MNRAS.447.4004R}).



%\begin{itemize}
%\item{Cold core in the evolutionary context of the ISM}
%\item{Cold temperature chemistry: COMs ?}
%\item{Status of present work on chemistry in dense cores (TMC1, B1 ?), deep integrations of a few selected sources}
%\end{itemize}
%Need of systematic studies -> Survey
  
  
  
  
  
  
  
  
  
  
  
  
  
  
  
  