\section{Scientific context} 

Molecules are present everywhere in the interstellar medium, even in the diffuse medium where the UV field was believed to destroy them (see for instance \cite{2012ApJ...753L..28L}). In the densest and coldest parts of the ISM (i.e. cold cores) shielded from the interstellar UV field, many molecules are formed and can survive. The formation of these species in the gas-phase has been studied for a long time now. Chemical networks have been built to index all individual reactions and associated rate coefficients that play a role in the determination of the abundances of molecules observed. It was first believed that ion-neutral reactions were the driving mechanism of this chemistry. But with the development of experimental methods allowing for measurements of rate coefficients at low low temperatures in the laboratory, it appeared clear that many neutral-neutral reactions would be crucial at the lowest temperatures (\cite{2010SSRv..156...13W}, \cite{Smith_2011}). Since 2009, with a group of chemists, we have been working on gas-phase networks in order to propose to the community the most complete and accurate data to model the chemistry in the ISM. Up to now, several families have been revisited: 
\begin{itemize}
\item{Nitrogen bearing species (\cite{2013PCCP...1513888D}, \cite{2012PNAS..10910233D}, \cite{2014MNRAS.443..398L}, \cite{2013arXiv1310.4350W}. We in particular studied the reactions involved in the chemistry of N2, HCN and HNC. 




\begin{itemize}
\item{Cold core in the evolutionary context of the ISM}
\item{Cold temperature chemistry: COMs ?}
\item{Status of present work on chemistry in dense cores (TMC1, B1 ?), deep integrations of a few selected sources}
\end{itemize}
Need of systematic studies -> Survey
  
  
  
  
  
  
  