\section{Technical justification}
 Within cold dense cores the linewidths are expected to be narrow, of the order of 1 km/s. Using the FTS in the fine mode, with a channel width of 50kH, the velocity resolution will be higher than 0.19 km/s over the full observed bandwidth, resolving the line profiles with at least 5 channels per FWHM.

With 1h integration time per frequency channel of 50kHz, a uniform sensitivity survey can be achieved with a median rms of 30mK over the full 3mm band and 35mK over the lower 36GHz of the 2mm band. 

A 36Ghz bandwidth can be covered uniformly, in dual polarization, with 10 tunings using the FTS in its 50kHz narrow mode and one of the sideband. Nevertheless because of the limited band rejection (particularly at 2mm) care must be ensured to observe each sky frequency with at least 2 different LO tunings. This is why we propose to use 18 tunings shifted by half an FTS unit bandwidth each observed for 30 minutes integration time (see Fig~\ref{fig:setup})

To reduce tuning overheads, the 5 sources have been chosen nearby on the sky so that a common frequency setup and pointing source can be used. The observing strategy will thus be: 
\begin{program}
\end{program}
\noindent for i 1 to 18:\\
\indent  tune to LO frequency i\\
\indent  for j 1 to 5\\
\indent pointing\\
\indent observe source j (calibration every 10-15 min)

The total observing time is thus $18\times0.5\times5=45$ hours. Accounting for 15\% overhead (tuning, pointing and focus). The total telescope time is 52h.
   
\paragraph{Public data policy:} Before the end of the IRAM proprietary time, we will provide publicly the full \emph{reduced} dataset to the community both on the project's website and the CDS.
  
  
  
  
  