\section{Scientific objectives}

The main goal of this proposal is to make a chemical inventory of a sample of cold cores. Cold cores are the primitive material to form stars and planets. Constraining their chemical composition and its evolution during their collapse is a key in the understanding of our origin. Dense cores also very often test cases for astrochemical models because of their relative simple physical structures. While comparing chemical model predictions with observed abundances, it is necessary to consider a significant number of molecules \cite{Wakelam_2006,2013ChRv..113.8710A}. The only two sources that have been extensively studied and for which more than 20 molecular abundances have been derived are TMC-1 (Cyanopolyyne Peak) and L134N (N) (see \cite{2013ChRv..113.8710A}). These two cold cores present some obvious differences such as larger abundance of carbon chains in TMC-1 (CP) compared to L134N (N). These differences have been attributed to a difference in the C/O elemental ratio \cite{1998ApJ...501..207T} or in the age of the cores \cite{2013ChRv..113.8710A}. There are however many other parameters and processes to test and relaying on only two catalogs of observed abundances is not enough. In addition, cold cores can present strong chemical differences one from another \cite{2006FaDi..133...63B}.\\
In this proposal, we wish to make a spectral survey on a sample of 5 cold cores observed by Planck. The first direct objective will be to characterize their physical properties and the chemical abundances using radiative transfer models (???). Then using Bayesian inversion methods, we will evaluate the use of chemical codes to determine the cloud physical parameters (Gratier et al. in prep). Obtaining the chemical abundances of about 35 species in 5 cores, we will be able to study the variability of the chemistry among cores. There is a large number of questions that will be addressed if this proposal is accepted. Among them, the question on the isotopic fractionation of D, $^{13}C$ and $^{15}N$ at low temperature should give us some indication of the chemical processes, the ice formation at the very low temperatures and the time evolution of the clouds (\cite{2013A&A...551A..38P},\cite{2014prpl.conf..859C},\cite{2015A&A...576A..99R}). The measurements of carbon versus oxygen bearing species (such as CS/CO \cite{1997ApJ...482..285B} or CN/NO \cite{2014A&A...562A..83L}) may give us some constraints on the C/O elemental ratio in these objects and so the time scale of conversion of atoms to molecules. Finally, by addressing the chemical composition in dense clouds and their diversity, we will be able to understand the variability in the initial conditions of star and planet formation. 

  