\section{Scientific objectives}

Because of their relative structure simplicity, cold dense cores are very often used as test cases for astrochemical models. While comparing chemical model predictions with observed abundances, it is necessary to consider a significant number of molecules \cite{Wakelam_2006,2013ChRv..113.8710A} The only two sources that have been extensively studied and for which more than 20 molecular abundances have been derived are TMC-1 (Cyanopolyyne Peak) and L134N (N). The list of observed abundances have been listed in \cite{2013ChRv..113.8710A}. These two cold cores present some obvious differences such as larger abundances of carbon chains in TMC-1 (CP) compared to L134N (N). These differences have been attributed to a difference in the C/O elemental ratio \cite{1998ApJ...501..207T} or in the age of the cores \cite{2013ChRv..113.8710A}. There are however many other parameters and processes to test and relaying on only two catalogs of observed abundances is not enough. In addition, cold cores can present strong chemical differences one from another \cite{2006FaDi..133...63B} so validating chemical model on one or two sources is not statistically relevant.\\
We wish in this proposal to make a chemical inventory of other cold cores in order to increase the number of sources for which a sufficient number of molecules have been observed. In addition to providing to the community with the reduced data and computed abundances, we will test our current state-of-the-art gas-grain model Nautilus \cite{2015MNRAS.447.4004R} on a larger sample of cores.  Among the key questions we want to study are: What are the chemical processes involved in the formation of complex organic molecules in cold cores? What is the isotopic fractionation of D, $^{13}C$ and $^{15}N$ at low temperature? Which are the key parameters controlling the chemistry in these sources? 
What is the diversity in the core chemical composition and what will be the impact on the chemical composition of the resulting protostars and disks?  OTHERS AMONG WHAT PIERRE WROTE BELOW?

\begin{itemize}
%\item{Study the variability of chemistry in a sample of cold cores: (35 species in 5 cores)}
%\item{Study the isotopic fractionation at low temperature D, $^{13}C$, $^{15}N$} (cf Roueff et al. 2015)
\item{Evalutate the use of chemical codes  to determine physical parameters (density, temperature, grain temperature, extinction). Comparison with Planck data. Comparison with radiative transfer. Bayesian method of inversion}
%\item{Constrain low temperature carbon chain chemistry \chem{C_xH} and \chem{C_xH_2}
\end{itemize}
  
  
  
  
  
  
  
  
  
  
  
  